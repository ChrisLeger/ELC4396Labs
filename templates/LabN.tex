\documentclass{article}\usepackage{graphicx} % new way of doing eps files
\usepackage{listings} % nice code layout
\usepackage[usenames]{color} % color
\definecolor{listinggray}{gray}{0.9}
\definecolor{graphgray}{gray}{0.7}
\definecolor{ans}{rgb}{1,0,0}
\definecolor{blue}{rgb}{0,0,1}
% \Verilog{title}{label}{file}
\newcommand{\Verilog}[3]{
  \lstset{language=Verilog}
  \lstset{backgroundcolor=\color{listinggray},rulecolor=\color{blue}}
  \lstset{linewidth=\textwidth}
  \lstset{commentstyle=\textit, stringstyle=\upshape,showspaces=false}
  \lstset{frame=tb}
  \lstinputlisting[caption={#1},label={#2}]{#3}
}


\author{Christopher Leger}
\title{Lab1: Barrel Shifters}

\begin{document}
\maketitle

\section{Introduction}
The purpose of this was to test the various designs of barrel shifters which include the case design, where each possible shift is made, or the stage design, where shifts are made in powers of two to total the number of possible shifts. These shifters were design for both eight and sixteen bits to show the extension of these designs. There are also two different designs for choosing whether the shift is to the left or the right that are being shown throughout the lab. 
\section{Interface}
All of the shifters take in the amount of bits specified by the shifter, either eight or sixteen bits. Then the shifters require a value to describe how many shifts the user wants. For the eight bit shifters this value is three bits long and for the sixteen bit shifters this value is four bits long. The shifters then output the shifted bits with the same number of bits as the input. The next level up, the designs for choosing left or right shift both take in the same values as their associated shifters but also take in and extra one bit input that is the bit that decides which shift to use. The top module that is implemented on the board uses the switches to set the values passed to the shifters and the buttons for both the choice of left or right and the amount of shifts that are to be performed. 
\section{Design}
This is the internal design of the item.  Design description and explanation, including any relevant block diagrams, ASM charts, etc.
\section{Implementation}
The verilog code and explanations of why you implemented this way.  There are many ways to implement a given design in verilog.  For instance why choose a case statement or ifs?  Why did you trigger on a negedge verses any signal change?
\section{Test Bench Design}
This is where you discuss the test benches you wrote, and what they were designed to test.  You should discuss expected errors as well as random errors.  Be sure to include your verilog code.
\section{Simulation}
In this section you should show the results of your simulation, such as timing diagrams and explain any design issues you had to deal with before implementation on the FPGA.
\section{FPGA Realization and Final Verification}
Discuss the final implementation issues including the FPGA pin configuration files and programming the board.  Once programmed how did you verify functioning?
\section{Conclusions}
Overview the main points you want to stick in peoples minds and answer key questions you want to stick in peoples minds.  Did it work?  How well? What would you have done differently?  What did you learn?
\end{document} 